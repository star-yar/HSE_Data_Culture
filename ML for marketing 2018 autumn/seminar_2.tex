%!TEX TS-program = xelatex
\documentclass[12pt, a4paper, oneside]{article}

\usepackage{amsmath,amsfonts,amssymb,amsthm,mathtools}  % пакеты для математики

\usepackage[utf8]{inputenc} % задание utf8 кодировки исходного tex файла
\usepackage[british,russian]{babel} % выбор языка для документа

\usepackage{fontspec}         % пакет для подгрузки шрифтов
\setmainfont{Helvetica}   % задаёт основной шрифт документа

\usepackage{unicode-math}     % пакет для установки математического шрифта
\setmathfont{Neo Euler}      % шрифт для математики
% \setmathfont[math-style=ISO]{Asana Math}
% Можно делать смену начертания с помощью разных стилей

% Конкретный символ из конкретного шрифта
% \setmathfont[range=\int]{Neo Euler}

%%%%%%%%%% Работа с картинками %%%%%%%%%
\usepackage{graphicx}                  % Для вставки рисунков
\usepackage{graphics}
\graphicspath{{images/}{pictures/}}    % можно указать папки с картинками
\usepackage{wrapfig}                   % Обтекание рисунков и таблиц текстом

%%%%%%%%%%%%%%%%%%%%%%%% Графики и рисование %%%%%%%%%%%%%%%%%%%%%%%%%%%%%%%%%
\usepackage{tikz, pgfplots}  % язык для рисования графики из latex'a

%%%%%%%%%% Гиперссылки %%%%%%%%%%
\usepackage{xcolor}              % разные цвета

\usepackage{hyperref}
\hypersetup{
	unicode=true,           % позволяет использовать юникодные символы
	colorlinks=true,       	% true - цветные ссылки, false - ссылки в рамках
	urlcolor=blue,          % цвет ссылки на url
	linkcolor=red,          % внутренние ссылки
	citecolor=green,        % на библиографию
	pdfnewwindow=true,      % при щелчке в pdf на ссылку откроется новый pdf
	breaklinks              % если ссылка не умещается в одну строку, разбивать ли ее на две части?
}


\usepackage{todonotes} % для вставки в документ заметок о том, что осталось сделать
% \todo{Здесь надо коэффициенты исправить}
% \missingfigure{Здесь будет Последний день Помпеи}
% \listoftodos --- печатает все поставленные \todo'шки

\usepackage[paper=a4paper, top=20mm, bottom=15mm,left=20mm,right=15mm]{geometry}
\usepackage{indentfirst}       % установка отступа в первом абзаце главы

\usepackage{setspace}
\setstretch{1.15}  % Межстрочный интервал
\setlength{\parskip}{4mm}   % Расстояние между абзацами
% Разные длины в латехе https://en.wikibooks.org/wiki/LaTeX/Lengths


\usepackage{xcolor} % Enabling mixing colors and color's call by 'svgnames'

\definecolor{MyColor1}{rgb}{0.2,0.4,0.6} %mix personal color
\newcommand{\textb}{\color{Black} \usefont{OT1}{lmss}{m}{n}}
\newcommand{\blue}{\color{MyColor1} \usefont{OT1}{lmss}{m}{n}}
\newcommand{\blueb}{\color{MyColor1} \usefont{OT1}{lmss}{b}{n}}
\newcommand{\red}{\color{LightCoral} \usefont{OT1}{lmss}{m}{n}}
\newcommand{\green}{\color{Turquoise} \usefont{OT1}{lmss}{m}{n}}

\usepackage{titlesec}
\usepackage{sectsty}
%%%%%%%%%%%%%%%%%%%%%%%%
%set section/subsections HEADINGS font and color
\sectionfont{\color{MyColor1}}  % sets colour of sections
\subsectionfont{\color{MyColor1}}  % sets colour of sections

%set section enumerator to arabic number (see footnotes markings alternatives)
\renewcommand\thesection{\arabic{section}.} %define sections numbering
\renewcommand\thesubsection{\thesection\arabic{subsection}} %subsec.num.

%define new section style
\newcommand{\mysection}{
	\titleformat{\section} [runin] {\usefont{OT1}{lmss}{b}{n}\color{MyColor1}} 
	{\thesection} {3pt} {} } 


%	CAPTIONS
\usepackage{caption}
\usepackage{subcaption}
%%%%%%%%%%%%%%%%%%%%%%%%
\captionsetup[figure]{labelfont={color=Turquoise}}

\pagestyle{empty}

\begin{document}

\section*{Семинар 3-4:  привлечение клиентов и классификация}

\subsection*{Задача 0}

Ликбез!  Со всей глубиной и знанием дела дайте ответы на следующие вопросы: 

\begin{itemize}
\item  Почему нумерация в этом семинаре начинается с нуля? 
\item  Что такое машинное обучение и что оно позволяет делать? 
\item  Что такое обучение без учителя и чем оно отличается от обучения с учителем? 
\item  Чем задача классификации отличается от задачи кластеризации? 
\item  Чем задача классификации отличается от задачи регрессии? 
\item  Как оценить модель? Что для этого нужно? 
\item Что такое метрика качества модели? Какие метрики вы знаете? Как правильно измерить качество модели? 
\item Что такое кросс-валидация?  Как объяснить это бабушке? 
\end{itemize}

В смысле не знаете? Мы целый модуль этим занимались!  Слабо дать ответ на каждый вопрос с помощью одного ёмкого слова? 

\subsection*{Задача 1}

Есть четыре человека и Вася. У всех профили с кинопоиска. 

Фильмы какого жанра скорее всего понравятся Васе, если доверить выбор методу одного ближайшего соседа? 
А если выбирать по двум ближайшим соседям? (АХТУНГ!)
А если выбирать по трём соседям? 
Почему нельзя брать очень много соседей? 

Дайте ответы на вопросы выше, используя евклидово расстояние. 



\subsection*{Задача 2}

На плоскости расположены колонии рыжих и чёрных муравьёв. Рыжих колоний три и они имеют координаты $(-1, -1)$, $(1, 1)$ и $(3, 3)$. Чёрных колоний тоже три и они имеют координаты $(2, 2)$, $(4, 4)$ и $(6, 6)$.

\begin{enumerate}
	\item Чем KNN отличается от K-means? 
	\item Поделите плоскость на «зоны влияния» рыжих и чёрных используя метод одного ближайшего соседа.
	\item Поделите плоскость на «зоны влияния» рыжих и чёрных используя метод трёх ближайших соседей.
	\item С помощью кросс-валидации с выкидыванием отдельных наблюдений выберите оптимальное число соседей $k$ перебрав $k \in \{1, 3, 5\}$. Целевой функцией является количество несовпадающих прогнозов.
\end{enumerate}

\subsection*{Задача 3}

Обучаем дерево своими руками. Правило разбиение вершин: минимизация числа допущенных ошибок. Правило прогнозирования в каждой вершине: в качестве прогноза выдаем тот класс, представителей которого в вершине больше. 



\subsection*{Задача 4}

По данной диаграмме рассеяния постройте классификационное дерево для зависимой переменной $y$:

	\begin{center}
		\begin{tikzpicture}[scale = 0.015]
		\input{images/tree_scatter_data.tikz}
		\end{tikzpicture}
	\end{center}


\section{Ещё задачи} 

\subsection*{Задача 5} 

Ниже изображены разделяющие поверхности для задачи бинарной классификации, соответствующие решающим деревьям разной глубины. Какое из изображений соответствует наиболее глубокому дереву?

\begin{center}
	\includegraphics[scale=0.7]{trees.png}
\end{center}


\subsection*{Задача 6} 

Рассмотрим обучающую выборку для прогнозирования $y$ с помощью $x$ и $z$:

\begin{center}
\begin{tabular}{c|c|c}
	\hline
	$y_i$ & $x_i$ & $z_i$ \\
\hline
	$y_1$ & $1$ & $2$ \\
	$y_2$ & $1$ & $2$ \\
	$y_3$ & $2$ & $2$ \\
	$y_4$ & $2$ & $1$\\
	$y_5$ & $2$ & $1$ \\
	$y_6$ & $2$ & $1$ \\
	$y_7$ & $2$ & $1$ \\
\end{tabular}
\end{center}

Будем называть деревья разными, если они выдают разные прогнозы на обучающей выборке.
Сколько существует разных классификационных деревьев  для данного набора данных?

\subsection*{Задача 7} 

Решите первую и вторую задачи, используя манхеттенское расстояние вместо евклидова. Можно ли подбирать метрику для подсчёта расстояния с помощью кросс-валидации, как мы делали это с параметром $k$? 

\end{document}
