\documentclass[pdftex, 12pt, a4paper]{article}

\usepackage[british,russian]{babel} % выбор языка для документа
\usepackage[utf8]{inputenc} % задание utf8 кодировки исходного tex файла
\usepackage[X2,T2A]{fontenc}

\usepackage{amsmath,amsfonts,amssymb,amsthm} % AMS

\usepackage[pdftex,unicode,colorlinks=true,urlcolor=blue,hyperindex,breaklinks]{hyperref}  %гиперссылки


%%% Работа с картинками
\usepackage{graphicx}  % Для вставки рисунков
\usepackage{wrapfig} % Обтекание рисунков и таблиц текстом
\usepackage{xcolor}
\usepackage{colortbl}
\usepackage{pgf,tikz}
\usepackage{mathrsfs}
\usetikzlibrary{arrows}

%%% Работа с таблицами
\usepackage{tabularx,tabulary}
\usepackage{longtable}  % Длинные таблицы
\usepackage{multirow} % Слияние строк в таблице


\usepackage{todonotes} % для вставки в документ заметок о том, что осталось сделать
% \todo{Здесь надо коэффициенты исправить}
% \missingfigure{Здесь будет Последний день Помпеи}
% \listoftodos --- печатает все поставленные \todo'шки


\oddsidemargin=0pt
\topmargin=0pt
\textwidth=16cm
\pagestyle{empty}

\DeclareMathOperator{\Corr}{Corr}
\DeclareMathOperator{\sCorr}{sCorr}
\DeclareMathOperator{\sCov}{sCov}
\DeclareMathOperator{\sVar}{sVar}
\DeclareMathOperator{\Cov}{Cov}
\DeclareMathOperator{\Var}{Var}

\def \RR{\mbb R}
\def \NN{\mbb N}
\def \ZZ{\mbb Z}
\def \PP{\mbb{P}}
\newcommand{\E}{\mathbb{E}}
\def \QQ{\mbb Q}

\def\s{\sigma}
\def \a{\alpha}
\def \b{\beta}
\def \t{\tau}
\def \dt{\delta}
\newcommand{\e}{\varepsilon}
\def \ga{\gamma}
\def \kp{\varkappa}
\def \la{\lambda}
\def \sg{\sigma}
\def \sgm{\sigma}
\def \tt{\theta}
\def \Dt{\Delta}
\def \La{\Lambda}
\def \Sgm{\Sigma}
\def \Sg{\Sigma}
\def \Tt{\Theta}
\def \Om{\Omega}
\def \om{\omega}

% вместо горизонтальной делаем косую черточку в нестрогих неравенствах
\renewcommand{\le}{\leqslant}
\renewcommand{\ge}{\geqslant}
\renewcommand{\leq}{\leqslant}
\renewcommand{\geq}{\geqslant}

\newcommand{\II}{{\fontencoding{X2}\selectfont\CYRII}}   % I десятеричное (английская i неуместна)
\newcommand{\ii}{{\fontencoding{X2}\selectfont\cyrii}}   % i десятеричное
\newcommand{\EE}{{\fontencoding{X2}\selectfont\CYRYAT}}  % ЯТЬ
\newcommand{\ee}{{\fontencoding{X2}\selectfont\cyryat}}  % ять
\newcommand{\FF}{{\fontencoding{X2}\selectfont\CYROTLD}} % ФИТА
\newcommand{\ff}{{\fontencoding{X2}\selectfont\cyrotld}} % фита
\newcommand{\YY}{{\fontencoding{X2}\selectfont\CYRIZH}}  % ИЖИЦА
\newcommand{\yy}{{\fontencoding{X2}\selectfont\cyrizh}}  % ижица

\newcounter{nc}
\newcommand{\ex}[1]{%
\addtocounter{nc}{1}
\textcolor{blue}{Задача \arabic{nc}}

#1\\}

%\renewcommand{\labelitemi}{\asbuk{itemi}}
%\renewcommand{\labelitemii}{\asbuk{itemii}}

\usepackage{float}

\begin{document} % конец преамбулы, начало документа

\begin{problem}
	Бандерлог из Лога\footnote{деревня в Кадуйском районе Вологодской области} ведёт блог, любит считать логарифмы и оценивать логистические регрессии. С помощью нового алгоритма Бандерлог решил задачу классификации по трём наблюдениям и получил $b_i = \hat\P(y_i = 1|x_i)$.
	
	\begin{tabular}{cc}
		\toprule
		$y_i$ & $b_i$ \\
		\midrule
		1 & 0.7 \\
		-1 & 0.2 \\
		-1 & 0.3 \\
		\bottomrule
	\end{tabular}
	
	\begin{enumerate}
		\item Постройте ROC-кривую.
		\item Найдите площадь под ROC-кривой и индекс Джини.
		\item Постройте PR-кривую (кривая точность-полнота).
		\item Найдите площадь под PR-кривой.
		\item Как по-английски будет «бревно»?
	\end{enumerate}
	\begin{sol}
	\end{sol}
\end{problem}


\begin{problem}
	Бандерлог начинает все определения со слов «это доля правильных ответов»:
	\begin{enumerate}
		\item accuracy — это доля правильных ответов\ldots
		\item точность (precision) — это доля правильных ответов\ldots
		\item полнота (recall) — это доля правильных ответов\ldots
		\item TPR — это доля правильных ответов\ldots
	\end{enumerate}
	
	Закончите определения Бандерлога так, чтобы они были, хм, правильными.
	\begin{sol}
		\begin{enumerate}
			\item $\text{accuracy} = \frac{\text{TP} + \text{TN}}{\text{TP} +\text{FP} +\text{FN} +\text{TN}}$
			\item $\text{precision} = \frac{\text{TP}}{\text{TP} +\text{FP}}$
			\item $\text{recall} = \frac{\text{TP}}{\text{TP} +\text{FN}}$
			\item $\text{TPR} = \frac{\text{TP}}{\text{TP} +\text{FN}}$
		\end{enumerate}
	\end{sol}
\end{problem}


\begin{problem}
	Алгоритм бинарной классификации, придуманный Бандерлогом, выдаёт оценки вероятности $b_i = \hat\P(y_i=1 | x_i)$. Всего у Бандерлога 10000 наблюдений. Если ранжировать их по возрастанию $b_i$, то окажется что наблюдения с $y_i = 1$ занимают ровно места с  5501 по 5600.
	
	Найдите площадь по ROC-кривой и площадь под PR-кривой.
	\begin{sol}
	\end{sol}
\end{problem}






\end{document}
